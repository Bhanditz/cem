\subsection{{\tt imbalance}: Calculates several imbalance measures}\label{ss:imbalance}
\keyword{datagen}{imbalance}
\begin{Description}\relax
Calculates several imbalance measures for the original
and matched data sets
\end{Description}
\begin{Usage}
\begin{verbatim}
imbalance(group, data, drop=NULL, breaks = NULL, weights)
\end{verbatim}
\end{Usage}
\begin{Arguments}
\begin{ldescription}
\item[\code{group}] the group variable
\item[\code{data}] the data
\item[\code{drop}] a vector of variable names in the data frame to ignore
\item[\code{breaks}] a list of vectors of cutpoints used to calculate the L1
measure. See Details.
\item[\code{weights}] weights
\end{ldescription}
\end{Arguments}
\begin{Details}\relax
This function calculates several imbalance measures.
For numeric variables, the difference in means (under the column
\code{statistic}), the difference in quantiles and the L1 measure is
calculated. For categorical variables the L1 measure and the
Chi-squared distance (under column \code{statistic}) is calculated.
Column \code{type} reports either \code{(diff)} or \code{(Chi2)} to
indicate the type of statistic being calculated.

If \code{breaks} is not specified, the Scott automated bin calculation
is used (which coarsens less than Sturges, which used in
\code{\LinkA{cem}{cem}}).  Please refer to \code{\LinkA{cem}{cem}} help page. In
this case, breaks are used to calculate the L1 measure.

This function also calculate the global L1 imbalance measure. 
If \code{breaks} is missing, the default rule to calculate cutpoints
is the Scott's rule. 
See \code{\LinkA{L1.meas}{L1.meas}} help page for details.
\end{Details}
\begin{Value}
An object of class \code{imbalance} which is a list with the following
two elements
\begin{ldescription}
\item[\code{tab}] Table of imbalance measures
\item[\code{L1}] The global L1 measure of imbalance
\end{ldescription}
\end{Value}
\begin{Author}\relax
Stefano Iacus, Gary King, and Giuseppe Porro
\end{Author}
\begin{References}\relax
Stefano Iacus, Gary King, Giuseppe Porro, ``Matching for
Casual Inference Without Balance Checking,''
http://gking.harvard.edu/files/abs/cem-abs.shtml
\end{References}
\begin{Examples}
\begin{ExampleCode} 

data(LL)

todrop <- c("treated","re78")
   
imbalance(LL$treated, LL, drop=todrop)

# cem match: automatic bin choice
mat <- cem(treatment="treated", data=LL, drop="re78")
\end{ExampleCode}
\end{Examples}

