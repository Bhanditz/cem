\subsection{{\tt att}: Example of ATT estimation from CEM output}\label{ss:att}
\aliasA{plot.cem.att}{att}{plot.cem.att}
\keyword{multivariate}{att}
\begin{Description}\relax
An example of ATT estimation from CEM output
\end{Description}
\begin{Usage}
\begin{verbatim}
att(obj, formula, data, model="linear", extrapolate=FALSE, ntree=2000)
## S3 method for class 'cem.att':
plot(x, obj, data, vars=NULL,...)
\end{verbatim}
\end{Usage}
\begin{Arguments}
\begin{ldescription}
\item[\code{obj}] a \code{cem.atch} or \code{cem.match.list} object
\item[\code{formula}] a model formula. See Details.
\item[\code{data}] a single data.frame or a list of data.frame's in case of \code{cem.match.list}
\item[\code{model}] one model. See Details.
\item[\code{extrapolate}] extrapolate the CEM restriced estimate to the whole data. Default = \code{FALSE}.
\item[\code{ntree}] number of trees to generate in random forest model. Default = \code{2000}.
\item[\code{x}] the output from the  \code{att} function
\item[\code{vars}] a vector of variable names to be used in the parallel plots. By default all variables
involved in data matching are used.
\item[\code{...}] passed to the plot function.
\end{ldescription}
\end{Arguments}
\begin{Details}\relax
Argument \code{model} can be \code{lm, linear} for linear regression
model; \code{logit} for the the logistic model;
\code{lme, linear-RE} for the linear model with random effects. 
Also \code{rf, forest} for the randomforest algorithm.

If the outcome is \code{y} and the
treatment variable is \code{T}, then a \code{formula} like \code{y \textasciitilde{} T}
will produce the simplest estimate the ATT: with lm, it is just the
coefficient on \code{T}, which is the same as the difference in means,
weighted by CEM stratum size.  Users can add covariates to span any
remaining imbalance after the match, such as \code{y \textasciitilde{} T + age + sex},
to adjust for variables \code{age} and \code{sex}.

In the case of multiply imputed datasets, the model is applied to each
single matched data and the ATT and is the standard error estimated
using the standard formulas for combining results of multiply imputed
data.  

When \code{extrapolate} = \code{TRUE}, the estimate model is extrapolated
to the whole set of data.


There is a \code{print} method for the output of \code{att}. Specifying the
option \code{TRUE} in a \code{print} command gives complete output from the
estimated model when availalble.
\end{Details}
\begin{Value}
A matrix of estimates with their standard error, or a list in
the case of \code{cem.match.list}.
\end{Value}
\begin{Author}\relax
Stefano Iacus, Gary King, and Giuseppe Porro
\end{Author}
\begin{References}\relax
Stefano Iacus, Gary King, Giuseppe Porro, ``Matching for
Casual Inference Without Balance Checking,''
http://gking.harvard.edu/files/abs/cem-abs.shtml
\end{References}
\begin{Examples}
\begin{ExampleCode} 
data(LL)

# cem match: automatic bin choice
mat <- cem(treatment="treated",data=LL, drop="re78")
mat
mat$k2k

# ATT estimate
homo1 <- att(mat, re78~treated,  data=LL)
rand1 <- att(mat, re78~treated,  data=LL, model="linear-RE")
rf1 <- att(mat, re78~treated,  data=LL, model="rf")

homo2 <- att(mat, re78~treated,  data=LL, extra=TRUE)
rand2 <- att(mat, re78~treated,  data=LL, model="linear-RE", extra=TRUE)
rf2 <- att(mat, re78~treated,  data=LL, model="rf", extra=TRUE)

homo1
rand1
rf1

homo2
rand2
rf2

plot( homo1, mat, LL, vars=c("age","education","re74","re75"))
plot( rand1, mat, LL, vars=c("age","education","re74","re75"))
plot( rf1, mat, LL, vars=c("age","education","re74","re75"))

plot( homo2, mat, LL, vars=c("age","education","re74","re75"))
plot( rand2, mat, LL, vars=c("age","education","re74","re75"))
plot( rf2, mat, LL, vars=c("age","education","re74","re75"))

# reduce the match into k2k using euclidean distance within cem strata
mat2 <- k2k(mat, LL, "euclidean", 1)
mat2
mat2$k2k

# ATT estimate after k2k
att(mat2, re78~treated, data=LL)

# example with missing data
# using multiply imputated data
# we use Amelia for multiple imputation
if(require(Amelia)){
 data(LL)
 n <- dim(LL)[1]
 k <- dim(LL)[2]

# we generate missing values in 30
# randomly in one colum per row
 LL1 <- LL
 idx <- sample(1:n, .3*n)
 invisible(sapply(idx, function(x) LL1[x,sample(2:k,1)] <<- NA))

 imputed <- amelia(LL1) 
 imputed <- imputed$imputations[1:5]

 mat <- cem("treated", datalist=imputed, data=LL1, drop="re78")

 print(mat)
  
 att(mat, re78 ~ treated, data=imputed)
}
\end{ExampleCode}
\end{Examples}

